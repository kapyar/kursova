\documentclass[a4paper, 14pt]{extarticle}
\usepackage[T2A]{fontenc}
\usepackage[utf8]{inputenc}
\usepackage{amsmath}
\usepackage{lipsum}
\usepackage{geometry}
\usepackage{tabularx}

 \geometry{
a4paper,
total={210mm,297mm},
left=30mm,
right=15mm,
top=20mm,
bottom=20mm,
heightrounded,
}

\newcommand\textbox[1]{%
  \parbox{.333\textwidth}{#1}%
}


\usepackage[ukrainian,russian,english]{babel} 
\date{\vspace{-5ex}}

%configuring appearance
\linespread{1.3} % полуторный интервал
\frenchspacing


%page numbering
\usepackage{fancyhdr}
\pagestyle{fancy}
\fancyhf{}
\fancyhead[R]{\thepage}
\fancyheadoffset{0mm}
\fancyfootoffset{0mm}
\setlength{\headheight}{17pt}
\renewcommand{\headrulewidth}{0pt}
\renewcommand{\footrulewidth}{0pt}
\fancypagestyle{plain}{ 
    \fancyhf{}
    \rhead{\thepage}}
\setcounter{page}{1} % начать нумерацию страниц с №5
 
%custom commands
\newcommand{\Alpha}{\mathrm{A}}
\author{Ярослав Каплунський}


\begin{document}
\thispagestyle{empty}
\begin{center}
	Міністерство освіти і науки\\ НАЦІОНАЛЬНИЙ УНІВЕРСИТЕТ "КИЄВО-МОГИЛЯНСЬКА АКАДЕМІЯ"
	\\Факультет інформатики
	\\Кафедра математики
\end{center}
		
		
		\vspace{35mm}
		\begin{center}
			Курсова робота на тему :
		\end{center}


\begin{Large}
	\begin{center}
		\textbf{
			"Застосування теорії керованих марківських ланцюгів з неперервним 						часом і дисконтованим критерієм до задач обслуговування мереж"
		}
	\end{center}
		
\end{Large}

\vspace{35mm}


\begin{flushright}
	\begin{tabular}{@{}l@{}}
		Керівник курсової роботи \\
		Доцент Чорней Р.К.\\
		
		\noindent\rule{4cm}{0.4pt}\\
		(\textit{підпис})\\
		\noindent\rule{1cm}{0.4pt} квітня 2015 р.\\
		Виконав студент 4 року навчання ФІ\\
		Каплунський Я.О.\\
	
	\end{tabular}
\end{flushright}

\vfill

\begin{center}
Київ 2015
\end{center}

\newpage
%Кінець першої сторіночки
\begin{center}
	{\Large \textbf{Зміст}}
\end{center}



\noindent Вступ \hfill 3\\
\noindent 1 \ \  Частина теоретична \hfill 4\\
\noindent   1.1 Керована марківська модель з дискретним часом \hfill 4\\
\noindent  1.2 Керування напівмарківською моделлю\hfill 11\\
\noindent 2  \ \ Частина практична \hfill 15\\
\noindent   Висновок \hfill 17\\
\noindent Список використаної літератури \hfill 18\\


\newpage
%кінець змісту

\begin{center}
	{\Large \textbf{Вступ}}
\end{center}

Зі системами масового обслуговування ми стикаємося кожного дня. З розвитком інформаційних технологій, системи масового обслуговування стають складніші. Прикладами систем масового  обслуговування можуть бути: телефонні системи, обчислювальні центри, хостинги, магазини. \par 
	Оптимальне керування цими системами дозволяє побудувати алгоритм вибору стратегій для отримання максимального прибутку або ж мінімізації витрат, що є однією з основних задач економіки.\\
	
Об'єктом дослідження є системи масового обслуговування. Предметом дослідження є керування цими системами як марківськими процесами.\\Метою дослідження є побудова програмного застосування, яке допоможе знайти оптимальну стратегію для керування системою масового\\ обслуговування.\\


В практичній частині ми розглянемо модель системи масового\\ обслуговування, для якої побудуємо, за допомогою  ітераційного підходу,   оптимальну стратегію, в залежності від вхідних параметрів нашої системи.





\newpage
\begin{center}
	\begin{Large}\textbf{Частина теоретична}\end{Large}
\end{center}

\section{Керована марківська модель з дискретним часом}

Нехай $ X $ та $ A $ - деякі повні сепарабелні метричні простори, а $\alpha$ та $\upsilon$ - $\sigma$  - алгебри борелівських  підмножин $ X $ та $ A $ відповідно. Припустимо, що задано відображення $ F $ (багатозначна функція), яка ставить у відповідність кожному $ x \in X $ деяку непусту замкнену множину $A_x \subseteq  A  $, таким чином $ \Delta = \lbrace \left( x, a \right) : x \in X, a \in  A_x \rbrace$, (граф відображення $ F $) вимірний за Борелем в добутку просторів $ X \times A $.\\

Розглянемо керовану систему з дискретним часом в фазовому просторі $ X $. Якщо в момент часу $ n (n = 0,1,2...)$ система знаходиться 
 в стані $ x \in X $, то може бити прийняте одне з множини $A_x$ допустиме рішення.\

Нехай $x_n$ і $a_n$ $(n = 0,1,2...)$ - послідовності станів рішень відповідно. Основне  припущення відносно ймовірносної еволюції керованої системи буде полягати в тому, що 

$$ P\lbrace x_{n+1} \in B \ | \ x_0,a_0,...,x_n =x, \ a_n=a\rbrace = P(B\ |\ x,a),	\ \ (1.1)$$ де $ B \in \Alpha $, $P(B \ | \ x, a)$ - при фіксованих $ x,a ((x,a) \in \Delta)$, ймовірнісна міра $(X,\Alpha)$, при фіксованому $ B $, вимірна за Борелем функція на $\Delta$.\

З функцією керованої системи пов'язаний прибуток : якщо в стані $x \in X$ прийняте рішення  $ a 
\in A_x$, то середній очікуваний дохід за один крок  дорівнює $r(x,a)$. Припустимо, що функція $r(x,a)$ вимірна за Борелем та обмежена на $\Delta$. Будемо вважати, що сукупність $ \lbrace X,A,\lbrace A_x \rbrace,P,r \rbrace$ визначає керовану марківську модель з дискретним часом.\\

Допустима стратегія $ R $ для рівняння марківської моделі визначається як послідовність $\lbrace\pi_0,\pi_1,...,\pi_n...\rbrace$, де $\pi_n ( \bullet \ | \ x_0,a_0,\ldots,x_n)$  - ймовірнісна міра на $(A,\Upsilon)$, зосереджена на $A_x$, з вимірним образом, яка залежить від $h_n=(x_0,a_0,\ldots,x_n)$ - історія керування системою до моменту $n$.\\
$(h_n \in \underbrace{\Delta \times \Delta \times \ldots \times \Delta}_{\text{n}} \times X) \pi_n(\bullet \ | \ x_0,a_0,\ldots,x_n)$ задає випадкове правило вибору рішення $a_n$ на основі інформації з $ h_n $\\

Стратегія $ R $ називається марківською, якщо $\pi_n(\bullet \ | \ x_0,a_0,\ldots,x_n) = \pi_n(\bullet \ | \ x_n) (n = 0,1,2\ldots)$, і стаціонарною нерандомізованою, якщо міра $\pi(\bullet \ | \ x)$ виродженна для будь-якого $x \  \in X$. Зрозуміло, що стаціонарну нерандомізовану стратегію $R$ можна ототожнити  з борелівською функцією $R(x)$, яка відображає  $X$ в $A$, при чому $ \forall x \in X, R(x) \in  A_x$ і $\pi(R(x)\ | \ x) = 1$. Позначимо $R$ - клас усіх допустимих стратегій $R_0$ - клас стаціонарних нерандомізованих стратегій.\par
В загальному випадку не відомо чи існує принаймні одна допустима стратегія. Достатні умови непустоти $R_0$ і відповідно $R$, можна отримати з теореми вибору.\newline \par

\textit{Означення 1}. Відображення $F$ ставить у відповідність кожному $x \in X$ деяку непусту замкнену множину $F(x) \subseteq A $, називають відкрито-, замкнено-, борелівсько- вимірною, якщо $\lbrace x\ : \ F(x)\bigcap E  \neq \emptyset \rbrace \in \Alpha$, де $E$ - відповідно довільна відкрита, замкнена або борелівська множина в $A$.
\newline \par

\textit{Означення 2}. Замкнено - (відкрито-),  вимірне відображення $F$\\ називається \textit{напівнеперервним зверху(знизу)}, якщо $\forall \ E \subseteq A$, де  $E$ множина замкнена (відкрита), $\lbrace x\ : \ F(x)\bigcap E  \neq \emptyset \rbrace$ замкнена (відкрита).
\newline \par

\textit{Означення 3}. Відображення $F$ називається \textit{неперервним}, якщо воно напівнеперервне зверху та знизу одночасно.\newline\par

\textit{Означення 4}. Функція $f \ : \ X \rightarrow A$ називається  \textit{селектором відображення} $F$, якщо $f(x) \in F(x), x \in X$.\newline\par
\textit{Означення 5}. Функція $f \ : \ A\rightarrow R$, де $A$ - метричний простір, називається функцією Бера.\\
До класу Бера 0 відносяться всі неперервні функції.\\
До класу Бера 1 відносяться всі розривні функції, які є поточковою границею послідовності неперервних функцій.\\
До класу Бера n>0 відносять функції, які не належать жодному класу Бера m<n, але які можна подати як поточкову границю послідовності функцій класів m<n.
Нумерація класів Бера не обмежується натуральними числами, і може бути продовжена за допомогою трансфінітних чисел.\newline\newline



\begin{center}
	\textbf{\textit{Теорема вибору для напівнеперервних відображень}}\newline
	Нехай вимірне в сенсі означення  відображення  має вимірний за Борелем селектор.	
\end{center}
Якщо $A$ -компакт, то напівнеперевне відображення має селектор, який належить першому класу Бера.\par
Отже, для того, щоб клас $R_0$ був непустим достатньо задати вимірність відображення $F \ : \ x \rightarrow A_x, x \in X$, в сенсі означення 1. Можна показати, що вимірність відображення $F$ тягне за собою  вимірність за Борелем \\ множини $\Delta$.\par
В подальшому будемо припускати, що відображення $F \ : \ x \rightarrow A_x, x \in X$ вимірним. Зокрема, якщо $A_x \equiv A$, відображення  $F$, очевидно, вимірне  в класі $R_0$ співпадає з множиною всіх борелівських функцій, які відображають $X$ в $A$. \par
Вибір стратегії $R$ ми визначаємо випадковим процесом, який взагалі кажучи не є марківським, так як вибір в момент часу $n$ може залежати від історії $h_n$. Назвемо цей процес випадковим , який керує стратегією $R$. Якщо стратегія $R$ марківська (стаціонарна), то керування процесом також буде марківським (марківським однорідним).\\


\begin{center}
Визначимо наступні критерії якості управління 
\end{center}

\begin{enumerate}
\item $$\psi_R(x,\beta) = M_x^R\sum_{n=0}^{\infty}\beta^nr(x_n,a_n),$$ де $0<\beta<1$. Під $M_x^R$ ми розуміємо умовне математичне очікування, яке відповідає процесу, керованому стратегією $R$ при умові $x_0 = x$.\par
Будемо вважати стратегію $R^*$ оптимальною відносно цього критерію, якщо $$ \psi_{R^*}(x,\beta) = \sup_{R \in \Re} \psi_R(x,\beta), \ x \in X$$\newline
назвемо цей критерій $\psi$ - критерієм, а стратегію $R^*$ $\psi$ - оптимальною.

\item $$\varphi_{R^*}(x) = lim_{n \rightarrow \infty} \inf \frac{1}{n+1}M_x^R\sum_{k=0}^nr(x_k,a_k),$$ назвемо цей критерій $\varphi$ - критерієм, а стратегію $R^*$, для якої \newline
	
 $\varphi_{R^*}(x) = \sup_{R \in \Re} \varphi_r(x), x \in X$, $\varphi$ - оптимальною
\end{enumerate}   


\begin{center}
	\textbf{\textit{Теорема про достатні умови існування $\varphi$ - оптимальної стратегії, яка належить $R_0$ класу }}\newline
\end{center}

\textbf{Теорема 1.1}\newline
Нехай існує обмежені борелівські функції $g(x)$ та $\upsilon(x)$ на $X$, такі, що $$ g(x)\geq \sup_{a \in A_x} \int g(y) P(dy \ | \ x,a), \ x \in X, $$
$$g(x)+\upsilon(x) \geq \sup_{a \in A_x} \lbrace r(x,a)+ \int \upsilon(y)P(dy \ | \ x,a) \rbrace, x \in X$$\newline
Тоді 
$$ \sup_{R \in \Re} \varphi_R(x) \leq g(x), x \in X $$
При цьому для деякої стратегії $R^*\in R_0$
$$g(x) = \int g(y)P(dy \ | \ x, R^*(x)), x \ in X$$
$$g(x) + \upsilon(x) = r(x,R^*(x))+ \int \upsilon(y) P(dy \ | \ x, R^*(x)), x \in X,$$
то $R^*$ $\varphi$ - оптимальною и \newline 
$\varphi_{R^*}(x)  \equiv g(x) ,$ як наслідок маємо наступну теорему.\newline

\textbf{Теорема 1.2}\newline
Нехай існує неперервна $g$ та обмежена борелівська функція $\upsilon(x)$ на $X$, такі, що 
$$ g+\vartheta(x)  = \max_{a \in A_x}\lbrace r(x,a)+ \int \vartheta(y) P(dy \ | \ x,a)\rbrace, x \in X \ \ (1.2) $$
тоді $$ \sup_{R \in \Re} \varphi_R (x) \leq g , x \in X. $$
Якщо для деяких стратегій $R^* \in R_0$\newline
$$ g +\varphi(x) = r(x, R^*(x))+\int \varphi(y)P(dy \ | \ x,R^*(x)), x \in X,  \ \ (1.3) $$
то $R^*$ $\varphi$ - оптимальною і 
$$ \varphi_{R^*}(x) \equiv g  $$
\textit{Зауваження 1}. 
Умовно обмежені функції $ g(x), \upsilon (x) $, можна замінити
$$ \sup_{(x,a) \in \Delta} \int |g(y)| P(dy \ | \ x,a) < \infty , \sup_{(x,a) \in \Delta} \int |\upsilon(y)| P(dy \ | \ x,a)<\infty$$

\textit{Зауваження 2}. В тому випадку, коли міра $P(\bullet | x,a)$ абсолютно неперервні відносно деякої міри $\lambda$ на $ (X,A)$ для теореми 1.2 достатньо виконання умов (1.1) та (1.2), відносно міри $\lambda$.\newline
Надалі співвідношення (1.1) будемо називати $\varphi$ - оптимальності.\newline

\textbf{Теорема 1.3}\newline
Нехай $A$ - компакт, відображення $F$ напівнеперервне зверху і виконується припущення (1.1). Тоді, якщо
\begin{enumerate}
\item функція $r(x,a)$ напівнеперервна зверху по $x, \ a ((x,a) \in \Delta)$
\item перехідна ймовірність $P(\bullet  \ | \ x,a)$ слабо неперервна по $x,a ((x,a) \in \Delta)$, то в $R_0$ існує $\varphi$ - оптимальна стратегія.
\end{enumerate}

\textbf{Теорема 1.4}\newline
Нехай $A$ зліченна, а множини $A_x (x \in X)$ скінченні і виконується припущення (1.1). Тоді в $R_0$ існує $\varphi $ - оптимальна стратегія. Ідея доведення полягає в наступному:\newline 
Визначимо оператор $\Upsilon$ в просторі $M(X)$ обмежених борелівських функцій на $X$, формулою
$$ \Upsilon \upsilon(x) = \sup_{a \in A_x} \lbrace r(x,a) + \int \upsilon(y)P^{'}(dy \ | \ x,a)  \rbrace, x \in X,  \ \ (1.4)$$ де 
$$ P^{'}(B \ | \ x,a) = P(B \ | \ x,a) - \mu(B), B \in A, \ (x,a) \in \Delta.  \ \ (1.5)$$\newline
Доводиться, що в теоремах 1.3, 1.4 оператор $\Upsilon$ переводить деякий повний метричний простір $S(X)\subseteq M(X)$  в себе і є  стискаючим. Як наслідок,
 $$\Upsilon \upsilon = \upsilon \ \ (1.6)$$ 
 має єдиний розвязок $\upsilon$ в $S(X) $. З (1.4) та (1.5) легко побачити, що \\ співвідношення (1.6) виражає рівняння $\varphi$ - оптимальності (1.2), де $ \vartheta = \upsilon   $ і $ g = \int \upsilon(x) \mu(dx)$. Потім доводиться, що відображення $F_{\upsilon}$ ставить кожному $x \in X$ множину 

 $$ F_{\upsilon}(x) = \lbrace a:a \in A_x, \Upsilon \upsilon(x) = r(x,a)+ \int \varphi(y)P^{'}(dy \ | \ x,a) \rbrace, $$ 
 
 вимірно, а отже за теоремою вибору, має вимірний за Борелем селектор  $R^{*}:X \rightarrow A$. Відповідна стратегія  $R^{*} \in R_0$ є $ \varphi $ - оптимальною за теоремою 1.2.\newline\newline
 

 
 \textbf{Теорема 1.5}\newline
Нехай $X $ та $A$ - компакти, відображення $F$ - неперервне та виконується припущення (1.6).\newline
 Тоді, якщо \newline
\begin{enumerate}
	\item функція $r(x,a)$ неперервна по $x,a ((x,a) \in \Delta$
	\item повна варіація міри $P(\bullet  \ | \ x,a)-P(\bullet  \ | \ x^{'},a^{'})$ прямує до нуля при $(x^{'},a^{'}) \rightarrow (x,a((x,a), \ (x^{'},a^{'}) \in \Delta)$, то в $R_0$ існує $\varphi$ -оптимальна стратегія $R^{*} : X \rightarrow A$ можна вибрати в 1 класі функцій Бера.
\end{enumerate}

 \textbf{Теорема 1.6}\newline
 Нехай $X \subseteq (- \infty, \infty), A $    зліченна, а множина $A_x (x \in X)$ скінченна та виконується співвідношення (1.6)\newline
Тоді, якщо\newline
\begin{enumerate}
\item функція $r(x,a)$ монотонно не спадає (не зростає) по $x$ на множині $X_a = \lbrace x \ : \ a\in A_x \rbrace \forall a \in A$
\item оператор $P_a \upsilon(x) = \int \upsilon(y)P(dy \ | \ x,a) \forall a \in A$ переводить монотонні неспадні обмеженні функції на $X$ в монотонно неспадні на $X_a$
\item множини $A_x$ монотонно не спадають (не зростають) за включенням
\item перехідна ймовірність $P(\bullet \ | \ x,a)$ абсолютно неперервні відносно міри Лебега, то в $R_0$ існують $\varphi$ - оптимальні стратегія. 
\end{enumerate}


 \textbf{Теорема 1.7}\newline
 Нехай $X$ зліченна, $A$ - компакт і виконується припущення (1.6). Тоді, якщо
 \begin{enumerate}
 \item функція $r(x,a)$ неперервна по $a$ на множині $A_x \forall x \in X$ 
 \item $ \sum_{y \in X} |P( \lbrace y \rbrace \ | \ x,a) \ - \ P(\lbrace y \rbrace | x,a^{'})| \rightarrow 0 \ $ при $a^{'}\rightarrow a \ (a,a^{'} \in A_x) \ \forall x \in X$, то в $R_0$ існує $\varphi$ - оптимальна стратегія
 \end{enumerate}
 \newpage
 

 \section{Керування напівмарківською моделлю}
\par Ця модель є  узагальненням керованої  марківскої моделі з дискретним часом та охоплює широке коло важливих задач дослідження операцій.
Нехай, як і раніше, $X$ та $A$ - деякі повні сепарабельні метричні простори, $\alpha$ та $\Upsilon$ - $\sigma$ - алгебри борелівських підмножин $X$ та $A$ відповідно, $F$ - вимірне відображення, яке ставить у відповідність кожному $x \in X$ деяку не пусту замкнену множину $A_x \subseteq A$.\par 
Розглянемо систему з фазовим простором $X$і множиною рішень $A$. В стані $x \in X$ допустимими є рішення з множини $A_x$. Прийняття рішення відбувається. Якщо в стані $x \in X$ прийняте рішення $a \in A_x$, то\newline
\begin{enumerate}
\item наступний стан системи обирається відповідно перехідної ймовірності $P(\bullet \ | \ x,a);$
\item  при умові, що наступний стан системи є $y \in  X$, час перебування в $x$ є випадковою величиною з функцією розподілу $\Phi(\bullet \ | \ x,a,y)$. $P(\bullet \ | \ x,a);$ та $\Phi(\bullet \ | \ x,a,y)$ за припущення є вимірними за Борелем функціями на $\Delta$ та $\Delta \times X$ відповідно.
\end{enumerate}
\par З функціонуванням системи пов'язаний дохід: якщо в стані $x \in X$ прийняте рішення $a \in A_x$ і час, виконання в стані $x$ дорівнює $t$, то очікуваний дохід за час $s \leq t$ дорівнює $r(s \ | \ x,a)$. Функція $r(s \ | \ x,a)$ має бути вимірною за Борелем на $ [0; \infty) \times \Delta.$\par 

Позначимо $x_n$ стан системи після $n$ - го переходу, $a_n$ - прийняте рішення, а $\tau_{n}$ час перебування в цьому станні $n = 1,2,3,\ldots$. Допустиму стратегію $R$ для керованої системи визначимо, як послідовність $\{ \pi_0, \pi_1, \ldots  \}$, де $\pi_n(\bullet \ | \ h_n)$ - ймовірнісна міра на $(A,\Upsilon)$, зосередженна на $A_{x_n}$ і вимірно залежить від $h_n = (x_0,a_0,\tau_0,\ldots , x_{n-1}, a_{n-1}, \tau_{n-1}, x_n)$ - історії керування системи до моменту $n$ - го переходу. $\pi_{n} (\bullet \ | \ h_n)$ задає рандомізоване правило вибору рішення $a_n$ на основі інформації $h_n$.\par 
 

Стратегія $R$ називається \textit{марківською}, якщо $\pi_n(\bullet \ | \ h_n) = \pi_n(\bullet \ | \ x_n), \ \ (n = 1,2,3,\ldots)$\par 
Марківська стратегія називається \textit{стаціонарною}, якщо міра $\pi(\bullet \ | \  x)$\\ виродженна для будь-якого $x \in X$. Позначимо $R$ - клас усіх допустимих стратегій, що $R_0$ - клас стаціонарних нерандомізованих стратегій. Зрозуміло, що $R_0$ можна ототожнити з сукупністю усіх Борелівських функцій $R(x)$, які відображають $X$ в $A$, таких, що $R(x) \in A_x$, $\pi(R(x) \ | \ x) = 1, \ x \in X.$

Непустоту класу $R_0$, а відповідно і $R$ гарантує умова вимірності $F$.\par 
Вибір стратегії $R$ ми визначимо як випадковий процес, який назвемо процесом, що керує стратегією $R$.\par 
Якщо стратегія $R$ стаціонарна, то керований процес є \textit{напівмарківським} процесом. В якості критерія оптимальності керування розглянемо наступний функціонал

$$ \varphi_{R^{*}}(x) = \lim_{n \rightarrow \infty} \inf \frac{ M_x^R \sum_{k=0}^n r( \tau_k | x_k,a_k)   }{M_x^R \sum_{k = 0}^{n} \tau_k}  \ \ (2.1),$$
 
де $M_x^R$ є символом математичного очікування, відповідного процесу, який керує стратегією $R$, при умові $x_0 = x$.\newline\par 
Назвемо стратегію $R^*$ оптимальною, якщо 

$$ \varphi_{R^*}(x) = \sup_{R \in \Re} \varphi_R(x), \ x \in X. $$
Позначимо
$$ \tau(x,a) = \int_X \int_0^{\infty} t d\Phi(t|x,a,y)P(dy|x,a), $$
$$ r(x,a) = \int_X \int_0^{\infty} r(t|x,a) d\Phi(t|x,a,y)P(dy|x,a). $$
Будемо вважати, що $\tau(x,a)$ і $r(x,a)$ існують і скінченні для всіх $(x,a) \in \Delta$ і $|r(x,a)| \leq K < \infty, (x,a) \in \Delta$. Очевидно, що критерій (2.1) залежить лише від $P(\bullet|x,a)$ і середніх характеристик $\tau(x,a), \ r(x,a)$. Тому не втрачаючи загальності, ми обмежимось розглядом процесів, для яких

\[ \Phi(t|x,a,y) = \left\{ 
  \begin{array}{l l}
    1  \quad \text{ $t \geq \tau(x,a)$}\\
    0  \quad \text{$t<\tau(k,a)$}
\end{array} \right.\]

\[ r(t|x,a) = \left\{ 
  \begin{array}{l l}
    r(x,a)  \quad \text{ $t \geq \tau(x,a)$}\\
    0  \quad \text{$t<\tau(k,a)$}
  \end{array} \right.\]
  
Будемо говорити, що сукупність $\{  X,A, \{ A_x \},P,r,\tau \},$ визначає керовану напівмарківську модель.\par
В попередньому розділі керування напівмарківською моделлю \\розглядалось у скінченних просторах $X$ та $A$. Попередньо ми довели що існує $\varphi$ - оптимальна стратегія, яка належить класу $R_0$, для будь-якої стаціонарної стратегії, всі стани якої $x \in X$ належать одному ергодичному класу відповідних вкладених марківських ланцюгів. В цьому розділі ми будемо говорити, про існування оптимальних стратегій, що належать класу $R_0$. Для існування в класі $R_0$ оптимальних стратегій нам потрібні наступні теореми:
\newline

 \textbf{Теорема 2.1}\newline
 Нехай виконується припущення (2.1) $$ \tau (x,a) \geq l > 0, (x,a) \in \Delta$$
 існує стала $g$  і обмежена борелівська функція $\upsilon(x)$ на $X$, такі, що:
 $$\upsilon((x) = \max_{a \in A_x} \{r(x,a) +\int \upsilon(y)P(dy|x,a) -g \tau(x,a)\}, x \in X  \ (2.2) $$ тоді
 $ \sup_{R \in \Re}\varphi_R(x) \leq g , x \in X$. \newline
 Якщо при цьому, для деякої стратегії $R^* \in R_0$
 
 $$ \upsilon(x) = r(x,R^*(x)) + \int \upsilon(y)P(dy|x,R^*(x)) - g \tau(x,R^*(x)), x \in X,\ (2.3) $$
 то $R^*$ оптимальна та $\varphi_{R^*}(x) \equiv g $
 \newline
 Існування константи $g$ і функцій $\upsilon(x) \in M(X)$, що задовільняють рівняння оптимальності (2.2), можна довести за допомогою принципу стиснутих \\відображень.\newline
\par Зробимо наступні припущення: \newline
 $\tau(x,a) \leq L < \infty$, $(x,a) \in \Delta$,\newline 
 існує невід'ємна міра $\mu$ на $(X,\Alpha)$ з $\mu(X) > 0$, така, що    
 $\mu(B)\leq F(B|x,a)$, $(x,a) \in \Delta, B \in \Alpha$\newline
 
 
  \textbf{Теорема 2.2}
 
 Нехай $A$  - компактна, відображення $F$ напівнеперервне зверху і\\ виконується попереднє припущення. Тоді, якщо
 \begin{enumerate}
 	\item функція $r(x,a)$ напівнеперервна зверху, в $r(x,a)$, неперервна по  \\ $x,a  ((x,a) \in \Delta)$;
 	\item перехідна ймовірність $P(\bullet|x,a)$ слабко неперервна по \\ $x,a \ ((x,a) \in \Delta)$,
 \end{enumerate}
 то в $R_0$ існує оптимальна стратегія.\newline
 
  \textbf{Теорема 2.3}
  Нехай $A$ зліченна, а множина $A_x \ (x \in X)$ скінченні і виконуються попередні припущення. Тоді в $R_0$ існує оптимальна стратегія.
 
 
 
 
 
 
 
 
 
 
 
 
 
  \newpage
 %Опис практичного завдання
 \begin{center}
	\begin{Large}\textbf{Частина практична}\end{Large}
\end{center}
 
 
Розглянемо керовану однолінійну систему масового обслуговування з пуасонівським вхідним потоком і параметром $\lambda>0$ і постійним часом обслуговування $c>0$. Максимальна допустима довжина черги рівна $K>0$. Керування полягає в тому, що в момент замовлення приймається рішення, брати чи не брати його на обслуговування. Вимоги, прийняті на \\обслуговування, починають негайно  обслуговуватися, якщо прилад вільний і стає в чергу в іншому випадку. Вимоги, які не були прийняті на\\ обслуговування - втрачаються.\par 
За кожну прийняту на обслуговування вимогу система масового \\ обслуговування отримує дохід $d>0$ і несе втрати, пропорційні часу\\ перебування цієї вимоги в черзі з коефіцієнтом пропорційності $r>0$.\par Задача полягає в знаходженні стратегії прийняття вимог на \\ обслуговування, яка максимізує середній очікуваний дохід за одиницю часу при нескінченній тривалості функціонування системи.\par Оскільки рішення ухвалюється в момент надходження запиту, то для побудови математичної моделі системи природньо перейти до дискретного часу, фіксуючи стан системи тільки в ці моменти. Під станом системи в момент надходження $n$ -го вимоги (в момент часу $n$) ми будемо розуміти пару $(k,t)$, де $k$ - число запитів, які знаходяться в системі, а $t$ - час, який пройшов з початку обслуговування запита, який виконує пристрій в даний момент. Позначимо $a_0$ рішення "не брати запит в обробку", а $a_1$ - "брати".\\
Дана система, яка описує керовану марківську модель з дискретним часом, в якої:
$$ X = \lbrace (0,0) \cup \lbrace 1,2,3 \ldots,K+1 \rbrace \times [0,c) \rbrace$$
$$ A = \lbrace a_0, a_1 \rbrace ;$$

 
 \[ A_{(k,t)} = \left\{ 
  \begin{array}{l l}
    A  \quad \text{ при \  $(k,t) \in X, k \leq K$}\\
    \{ a_0 \}  \quad \text{при $k = K+1$}
  \end{array} \right.\]
 
 $$ P(\lbrace k \rbrace \times [t,\tau) \ | \ (k,t), a_0) = 1 - e^{ -\lambda( \tau - t)}, (k,t) \in X, k > 0, \ t < \tau \leq c $$ 
 
 $$ P(\{  k-i \} \times [0,\tau) \ | \ (k,t), a_0 ) \ = \ e^{- \lambda(ic - t)}[1- e^{- \lambda \tau}],(k,t) \in X \ k > 0, \ i=1,\ldots , $$ $$k-1 \ t < \tau \leq c$$
 
 
 $$ P(\{ 0,0 \} \ | \ (k,t),a_0) = e^{- \lambda(kc -t)}, (k,t) \in X  $$
 
 $$ P(B \ | \ (k,t),a_1) = P(B \ | \ (k+1,t),a_0),(k,t) \in X , \ k \leq K, $$
де $B$ довільна Борелівська множина з $X$,

\[ r((k,t),a_i) = \left\{ 
  \begin{array}{l l}
   0 & \quad \text{$ i=0, (k,t) \in X$}\\
   d-r(kc -t),  & \quad \text{$ i=1,(k,t) \in X, k \leq K $}
  \end{array} \right.\]
  
  Звідси випливає
  $$ P(\{(0,0)\} \ | \ (k,t),a) \geq e^{-\lambda (K+1)c} = \alpha >0, (k,t) \in X, a \in A_(k,t) $$ і, очевидно, що перехідна ймовірність задовільняє припущення (1.1) з мірою $\mu$ маси $\alpha$ , зосереджена в точці $(0,0)$. За теоремою 1.4 для цієї керованої моделі існує $\varphi$ - оптимальна стратегія $R^{*}$, що належить класу $R_0$. Рівняння $\varphi$ - оптимальності, що дозволяє знаходити $R^{*}$, можна вирішити методом послідовних наближень. 
  
  \[ R^* = \left\{ 
  \begin{array}{l l}
   0 & \quad \text{ при рішенні $a_0$}\\
   1,  & \quad \text{  при рішенні $a_1$}
  \end{array} \right.\]
 
 
 \newpage
 
 
\begin{center}
	{\Large \textbf{Висновок}}
\end{center}
\par Отже, у даній роботі був розглянутий ітераційний підхід для знаходження оптимальної стратегії, для однорідного стаціонарного потоку без післядії. Для керування системою масового обслуговування  з одним сервісом, \\
пуасонівським вхідним потоком, постійним часом обслуговування і скінченною чергою у дискретні моменти часу. 


\par Взагалі кажучи, найпростіші потоки дуже рідко зустрічаються на практиці, але багато потоків, що моделюються, можна розглядати як найпростіші.
Написана програма, дозволяє легко знайти оптимальну стратегію. Результати роботи програми можна переглянути в додатку.

\newpage

\begin{center}
	{\Large \textbf{Список використаної літератури}}
\end{center}

\begin{enumerate}
\itemГубенко Л. Г. Керовані марківські і полумарківські моделі і деякі задачі оптимізації стохастичних систем / Губенко Л.Г. Штатланд Е.С.

\item Ховард Р. А. Динамічне програмування і марківські процеси / Ховард Р. А. – Москва: «Рядянське радіо»,1964. – 193 с.



\end{enumerate}
 

%addition
\newpage
\begin{center}
	{\Large \textbf{Додаток}}
\end{center}
Код програми можна знайти тут -  https://github.com/kapyar/kursova \newline

\begin{tabularx}{\textwidth}{ |X|X|X|X|X|X|X }
  \hline
  k & $\lambda$ & t & d & r & s \\
  \hline
 	10  & 33  & 1  & 500 & 50 & 1  1    1    0    0    0    0    0    0    0   \\
  \hline
   10  & 2  & 3  & 4 & 5 & 1    0    0    0    0    0    0    0    0    0 \\
  \hline
   10  & 12  & 3  &8 &5 &  1    0    0    0    0    0    0    0    0    0            \\
  \hline
   10  & 22  &3  & 180 & 5 & 1    1    1    1    1    1    1    1    1    1  \\
  \hline
   10  & 12   & 3 & 150 & 12 & 1    0    0    0    0    0    0    0    0    0  \\
  \hline
\end{tabularx}\\

$k$ - кількість запитів\newline
$\lambda$ - інтенсивність потоку\newline
$t$ - час на виконання запиту\newline
$d$ - дохід від виконання запиту\newline
$r$ - витрати на виконання запиту\newline
$s$ - оптимальна стратегія\newline


	




		
\end{document}
